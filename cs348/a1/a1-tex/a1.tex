\documentclass[12pt]{article}
\usepackage[top=2cm, bottom=2cm, left=2cm, right=2cm]{geometry}
\usepackage{enumitem}
\usepackage{amssymb}
\usepackage{amsmath,amsthm}
\usepackage{hyperref}
\usepackage{tabularx}
\newcommand{\code}[1]{\texttt{#1}}

\usepackage{listings}
\lstset{breaklines=true}

\begin{document}
\begin{center}
\Large{Assignment 1}\\
\small{Daniel Que, 20479762}
\end{center}
\section*{Part A}
Note: \string^ replaced with "and" due to problems in math mode with \LaTeX.

\subsection*{Q1}
$\sigma_{p\_brand = "Brand#42" \hat{} p\_size >= 43 \hat{} p\_size <= 47}(part)$

\subsection*{Q2}
$\pi_{c\_custkey, c\_name, c\_phone}(\sigma_{c\_acctbal < 0 \text{ and } c\_nationkey = n\_nationkey}(customer \times (\sigma_{n\_name = "CANADA"}(nation)))$

\subsection*{Q3}
$(\rho (R (c\_phone \rightarrow phone, c\_acctbal \rightarrow acctbal), \pi_{c\_phone, c\_acctbal} (\sigma_{c\_acctbal > 9975} (customer)))\\
\cup\\
(\rho (R (s\_phone \rightarrow phone, s\_acctbal \rightarrow acctbal), \pi_{s\_phone, s\_acctbal} (\sigma_{s\_acctbal > 9500} (supplier)))
$

\subsection*{Q4}
$nation \times supplier$
condition: n\_nationkey = s_nationkey

$\sigma_{s1.s_suppkey != s2.s\_suppkey and s1.s_nationkey = s2.s_nationkey } (supplier s1 \times supplier s2)$

$where t1.s\_suppk where t1.s\_suppkey ~= t2.s\_suppkey and t1.s\_nationkey == t2.s\_nationkeyey ~= t2.s\_suppkey and t1.s\_nationkey == t2.s\_nationkey$

RA:\\

\section*{Part B}

\subsection*{Q1}
\begin{verbatim}
SELECT P_PARTKEY FROM PART P
WHERE P.P_BRAND = 'BRAND#42' AND P.P_SIZE >= 43 AND P.P_SIZE <= 47
;
\end{verbatim}

\subsection*{Q2}
\begin{verbatim}
SELECT C_CUSTKEY, C_NAME, C_PHONE
FROM CUSTOMER C, NATION N
WHERE C.C_ACCTBAL < 0 AND C.C_NATIONKEY = N.N_NATIONKEY AND N.N_NAME = "CANADA"
;
\end{verbatim}

\subsection*{Q3}
\begin{verbatim}
SELECT C_NAME, C_PHONE, C_ACCTBAL
FROM CUSTOMER C
WHERE C_ACCTBAL > 9975
UNION
SELECT S_NAME, S_PHONE, S_ACCTBAL
FROM SUPPLIER S
WHERE S_ACCTBAL > 9500
;
\end{verbatim}

\subsection*{Q4}
\begin{verbatim}
SELECT N_NAME
FROM SUPPLIER S, NATION N
WHERE S_NATIONKEY=N_NATIONKEY
GROUP BY S_NATIONKEY
HAVING COUNT(*)>=2
;
\end{verbatim}
SQL: To make this change in SQL is a simple modification. Simply change the '2' in "HAVING COUNT(*)>=2" to a '7'.

\subsection*{Q5}
\begin{verbatim}
SELECT R_NAME, N_NAME
FROM REGION R LEFT OUTER JOIN NATION N
ON R.R_REGIONKEY=N.N_REGIONKEY
ORDER BY R_NAME, N_NAME ASC
;
\end{verbatim}

\subsection*{Q6}
\begin{verbatim}
SELECT COUNT(*) AS TOTAL
FROM (SELECT O_CUSTKEY, COUNT(*) NUM_ORDERS
  FROM ORDERS O
  GROUP BY O.O_CUSTKEY
  HAVING NUM_ORDERS = 0) AS TEMP
;
\end{verbatim}

\subsection*{Q7}
\begin{verbatim}
SELECT N_NAME, C_NAME, MAX(C_ACCTBAL) AS LARGEST_BALANCE
FROM (SELECT N_NAME, C_NAME, C_ACCTBAL
  FROM CUSTOMER C, NATION N
  WHERE C.C_NATIONKEY=N.N_NATIONKEY) TEMP
GROUP BY N_NAME
;
\end{verbatim}
If the assumption that every nation contains at least one customer that has at least one order is taken away, the query will simply return the top customers for nations that do have at least one, and exclude the nations with no customers.

\subsection*{Q8}
\begin{verbatim}
SELECT N_NATIONKEY, N_NAME, COUNT(*), CAST(SUM(O_TOTALPRICE) AS DECIMAL(13,2)) AS CUMULATIVE_ORDER_PRICE
FROM ORDERS O, CUSTOMER C, NATION N
WHERE O.O_CUSTKEY=C.C_CUSTKEY and C_NATIONKEY=N_NATIONKEY
GROUP BY N_NATIONKEY
ORDER BY CUMULATIVE_ORDER_PRICE DESC
;
\end{verbatim}

\subsection*{Q9}
\begin{verbatim}
SELECT N_NAME, NUM_ORDERS
FROM
  (SELECT N_NAME, N_NATIONKEY, O_ORDERKEY, C_CUSTKEY, COUNT(*) AS NUM_ORDERS
  FROM ORDERS O, CUSTOMER C, NATION N
  WHERE O_CUSTKEY=C_CUSTKEY and C_NATIONKEY=N_NATIONKEY
  GROUP BY N_NATIONKEY) TEMP
GROUP BY N_NATIONKEY
HAVING TEMP.NUM_ORDERS < 
  (SELECT AVG(NUM_ORDERS)
  FROM
    (SELECT N_NAME, N_NATIONKEY, O_ORDERKEY, C_CUSTKEY, COUNT(*) AS NUM_ORDERS
    FROM ORDERS O, CUSTOMER C, NATION N
    WHERE O_CUSTKEY=C_CUSTKEY and C_NATIONKEY=N_NATIONKEY
    GROUP BY N_NATIONKEY) TEMP)
;
\end{verbatim}

\subsection*{Q10}
\begin{lstlisting}
SELECT mktsegment1.m_mktsegment, CAST(mktsegment1.total AS DECIMAL(13,2)) AS total_f, CAST(mktsegment2.total AS DECIMAL(13,2)) AS total_o, CAST(mktsegment3.total AS DECIMAL(13,2)) AS total_p
FROM 
  (SELECT m_mktsegment, c_name, o_orderstatus, count(*), sum(o_totalprice) AS total
  FROM customer c, orders o, (SELECT c_mktsegment AS m_mktsegment
    FROM customer c
    GROUP BY c_mktsegment) mktsegment
  WHERE c.c_mktsegment=m_mktsegment and c_custkey=o_custkey
  GROUP BY m_mktsegment, o_orderstatus) mktsegment1,
  (SELECT m_mktsegment, c_name, o_orderstatus, count(*), sum(o_totalprice) AS total
  FROM customer c, orders o, (SELECT c_mktsegment AS m_mktsegment
    FROM customer c
    GROUP BY c_mktsegment) mktsegment
  WHERE c.c_mktsegment=m_mktsegment and c_custkey=o_custkey
  GROUP BY m_mktsegment, o_orderstatus) mktsegment2,
  (SELECT m_mktsegment, c_name, o_orderstatus, count(*), sum(o_totalprice) AS total
  FROM customer c, orders o, (SELECT c_mktsegment AS m_mktsegment
    FROM customer c
    GROUP BY c_mktsegment) mktsegment
WHERE c.c_mktsegment=m_mktsegment and c_custkey=o_custkey
GROUP BY m_mktsegment, o_orderstatus) mktsegment3
WHERE mktsegment1.m_mktsegment=mktsegment2.m_mktsegment and mktsegment2.m_mktsegment=mktsegment3.m_mktsegment and mktsegment1.o_orderstatus="f" and mktsegment2.o_orderstatus="o" and mktsegment3.o_orderstatus="p"
ORDER BY mktsegment1.m_mktsegment ASC
;
\end{lstlisting}

\end{document}

